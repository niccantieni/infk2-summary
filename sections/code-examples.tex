% SECTION ====================================================================================
\section{Code Beispiele}
% ============================================================================================

% \begin{sectionbox}
\subsection{Beispiel: Levenshtein-Algorithmus}
\begin{lstlisting}[language=C++]
#include <string>
#include <vector>

unsigned Levenshtein(const std::string& x, 
                      const std::string& y){
  // D[n,m] = distance between x and y
  // D[i,j] = distance between strings x[1..i] and y[1..j]
  unsigned n = x.size();
  unsigned m = y.size();
  std::vector<std::vector<unsigned>> D(
      n+1,std::vector<unsigned>(m+1,0)
    );
  for (unsigned j = 0; j <=m; ++j){
    D[0][j] = j;
  }
  for (unsigned i = 1; i <= n; ++i){
    D[i][0] = i;
    for (unsigned j = 1; j <=m; ++j){
      // D[i,j] = min{ 
      // D[i-1,j-1] + d(x[i],y[j]);
      // D[i-1,j] + 1
      // D[i,j-1] + 1 }
      D[i][j] = std::min( {
        D[i-1][j-1] + (x[i-1]!=y[j-1]), 
        D[i][j-1] + 1, 
        D[i-1][j] + 1
      });
    }
  }
  return D[n][m];
}
\end{lstlisting}
% \end{sectionbox}

\subsection{Dictionary in C++}
\begin{lstlisting}[language=C++]
#include <unordered_map>

// Create an unordered_map of strings that map to strings
std::unordered_map<std::string, std::string> colours = {
  {"RED", "#FF0000"}, {"GREEN", "#00FF00"}
};
colours["BLUE"] = "#0000FF"; //Add element
std::cout << "hex of red: " << colours["RED"] << "\n";


auto search = colours.find("BLUE"); //iterator to object
if (search != colours.end()) {
    std::cout << "Found " << search->first << 
    " : " << search->second << '\n';
} else {
    std::cout << "Not found\n";
}

//iterate
for (const auto& entry : colours) {
  std::cout << entry.first << " : " << entry.second << ", ";
  //BLUE : #0000FF, RED : #FF0000, GREEN : #00FF00,
}
\end{lstlisting}

\subsection{Sortieralgorithmen}
\begin{lstlisting}[language=C++]
// Exchange the elements of A at positions i and j.
void swap(std::vector<int>& a, 
          unsigned int i, 
          unsigned int j) {
    std::swap(a[i], a[j]);
}

void bubbleSort(std::vector<int>& a, 
                unsigned int l, 
                unsigned int r) {
    for (unsigned int i = l+1; i < r; i++) {
        for (unsigned int j = 0; j < r-i-1; j++) {
            if (a[j] > a[j+1])
                swap(a, j, j+1);
        }
    }
}
void insertionSort(std::vector<int>& a, 
                   unsigned int l, 
                   unsigned int r) {
    for (unsigned int i = l+1; i < r; i++) {
        unsigned int j = i-1;
        while (j >= 0 && a[j] > a[j+1]) {
            swap(a, j, j+1);
            j -= 1;
        }
    }
}
void selectionSort(std::vector<int>& a, 
                   unsigned int l, 
                   unsigned int r) {
    for (unsigned int i = l; i < r; i++) {
        unsigned int minJ = i;
        for (unsigned int j = i+1; j < r; j++) {
            if (a[j] < a[minJ])
                minJ = j;
        }
        if (minJ != i)
            swap(a, i, minJ);
    }
}
void quickSort(std::vector<int>& a, 
               unsigned int l, 
               unsigned int r) {
    if (l < r) {
        // here the pivot is a[l]
        unsigned int i = l+1;
        unsigned int j = r;
        
        while (i < j) {
            while (i < j && a[i] <= a[l])
                i += 1;
            while (i <= j && a[j] >= a[l])
                j -= 1;
            if (i < j)
                swap(a, i, j);
        }
        
        swap(a, l, j);
        quickSort(a, l, j-1);
        quickSort(a, j+1, r);
    }
}
\end{lstlisting}

\begin{sectionbox}
  
%\newcommand{\N}{\mathbb{N}}
%\newcommand{\R}{\mathbb{R}}
%

$$
  \begin{array}{l | c | c }
    \textbf{Bubble Sort} & \text{lower bound} & \text{upper bound}\\ \hline
    \text{Comparisons} & \bigO(n^2) & \bigO(n^2)\\
    \text{Sequence} & \text{any} & \text{any} \\ \hline
    \text{Swaps} & 0 & \bigO(n^2)\\
    \text{Sequence} & 1,2,...,n & n,n-1,...,1  \\
  \end{array}
$$


$$
  \begin{array}{l | c | c }
    \textbf{Insertion Sort} & \text{lower bound} & \text{upper bound}\\ \hline
    \text{Comparisons} & \bigO(n) & \bigO(n^2)\\
    \text{Sequence} & 1,2,...,n & n,n-1,...,1 \\ \hline
    \text{Swaps} & 0 & \bigO(n^2)\\
    \text{Sequence} & 1,2,...,n & n,n-1,...,1  \\
  \end{array}
$$
\end{sectionbox}
\vspace{-4pt}
\begin{sectionbox}

$$
  \begin{array}{l | c | c }
    \textbf{Selection Sort} & \text{lower bound} & \text{upper bound}\\ \hline
    \text{Comparisons} & \bigO(n^2) & \bigO(n^2)\\
    \text{Sequence} & \text{any} & \text{any} \\ \hline
    \text{Swaps} & 0 & \bigO(n)\\
    \text{Sequence} & 1,2,...,n & n,n-1,...,1 ~~(**) \\
  \end{array}
$$


$$
  \begin{array}{l | c | c }
    \textbf{Quick Sort} & \text{lower bound} & \text{upper bound}\\ \hline
    \text{Comparisons} & \bigO(n\log(n))& \bigO(n^2) \\
    \text{Sequence} & (*) & 1,2,...,n \\ \hline
    \text{Swaps} & \bigO(n) & \bigO(n \log(n))  \\
    \text{Sequence} & 1,2,...,n & (*) \\
  \end{array}
$$


(*): It is not easy to write down a compact form. The sequence must be constructed such that every pivot halves the sorting range. For instance for $n=7$ a sequence is: $4,5,7,6,2,1,3$.

(**): Even more swaps, exactly $n-1$ and with that the highest possible count, selectionSort uses for the sequence $n, 1, 2, 3, ..., n-1$.
\end{sectionbox}