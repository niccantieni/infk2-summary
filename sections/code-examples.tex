% SECTION ====================================================================================
\section{Code Beispiele}
% ============================================================================================

\begin{sectionbox}
\subsection{Beispiel: Levenshtein-Algorithmus}
\begin{lstlisting}[language=Python]
def Levenshtein(x, y):
  # D[n,m] = distance between x and y
  # D[i,j] = distance between strings x[1..i] and y[1..j]
  n = len(x)
  m = len(y)
  D = [[0 for i in range(m+1)] for j in range(n+1)]
  for j in range(0,m+1):
    D[0][j] = j
  for i in range(1,n+1):
    D[i][0] = i
    for j in range(1,m+1):
      # D[i,j] = min{ 
      # D[i-1,j-1] + d(x[i],y[j]);
      # D[i-1,j] + 1
      # D[i,j-1] + 1 }
      q = D[i-1][j-1] 
      if x[i-1] != y[j-1]:
        q += 1
      q = min(q, D[i][j-1]+1)
      q = min(q, D[i-1][j]+1)
      D[i][j] = q
  return D[n][m]
\end{lstlisting}
\end{sectionbox}

\begin{sectionbox}
\subsection{Beispiel: Längste gemeinsame Teilfolge}
\begin{lstlisting}[language=Python]
import Data
import time

# compute longest ascending sequence for a point of the matrix
def LASR(A,L,y,x):
  if L[y][x] > 0:
    return L[y][x]
  maxLength = 0
  if x>0 and A[y][x] < A[y][x-1]: 
    maxLength = max(maxLength, LASR(A,L,y,x-1))
  if y>0 and A[y][x] < A[y-1][x]: 
    maxLength = max(maxLength, LASR(A,L,y-1,x))
  if y<len(A)-1 and A[y][x] < A[y+1][x]: 
    maxLength = max(maxLength, LASR(A,L,y+1,x))
  if x<len(A[y])-1 and A[y][x] < A[y][x+1]: 
    maxLength = max(maxLength, LASR(A,L,y,x+1))
  L[y][x] = maxLength + 1;
  return L[y][x]

# compute longest ascending sequence for each point of the matrix
def LAS(A):
  maxLength = 0
  L = [[0] * len(A[i]) for i in range(len(A))]
  for y in range(len(A)):
    for x in range(len(A[y])):
      L[y][x] = LASR(A,L,y,x)
      maxLength = max(maxLength, L[y][x])
  return maxLength,L

A = Data.get()

start = time.time()
(m,L) = LAS(A)
stop = time.time();

if len(A)<15 and len(A[0])<15:
  print("matrix a")
  Data.print_matrix(A)
  print("path lengths matrix")
  Data.print_matrix(L)
  
print("maximum length",m)
print("time:", stop-start, "s")
\end{lstlisting}\vspace{-6px}
\end{sectionbox}

\begin{sectionbox}
\subsection{Beispiel: Union Find}
\begin{lstlisting}[language=Python]
class Set:
  def __init__(self,n):
    self.a = [i for i in range(0,n)]
    self.g = [1] * n
    
  def union(self,i,j):
    i = self.find(i)
    j = self.find(j)
    if i == j:
      return False
    self.a[j] = i # j under i
    return True

  def find(self,i):
    while self.a[i] != i:
      i = self.a[i]
    return i
  
  def path_length(self,i):
    count = 1;
    while self.a[i] != i:
        i = self.a[i];
        count = count + 1;
    return count;
    
class SmallUnderLarge(Set):
  def union(self,i,j):
    i = self.find(i)
    j = self.find(j)
    if i == j:
      return False
    if self.g[i] < self.g[j]:
      i,j = j,i
    self.a[j] = i # j under i
    if self.g[i] == self.g[j]:
      self.g[i] = self.g[i] + 1
    return True

class ConsolidateFind(Set):
  def find(self,i):
    root = i
    while self.a[root] != root:
      root = self.a[root]
    while self.a[i] != root:
      next = self.a[i]
      self.a[i] = root
      i = next
    return root
 
  # recursive version
  def find_recursive(self,i):
    if self.a[i] == i:
      return i
    self.a[i] = self.find(self.a[i])
    return self.a[i]
\end{lstlisting}\vspace{-6px}
\end{sectionbox}

\begin{sectionbox}
\subsection{Beispiel: Dict Comprehension}
\begin{lstlisting}[language=Python]
accounts = {
    'Food' : { 'Amount' : 1242, 'Kind': 'Credit' },
    'Insurances' : { 'Amount' : 5547, 'Kind': 'Credit' },
    'Fun-Time' : { 'Amount' : 3978, 'Kind': 'Credit'},
    'Salary'   : { 'Amount': 14785, 'Kind': 'Debit' },
    'Jewelry' : { 'Amount': 14785, 'Kind': 'Debit' }
}

credit_accounts = { account : record['Amount']
    for account, record in accounts.items() 
    if record['Kind'] == 'Credit'   }  # create your dict here
\end{lstlisting}\vspace{-6px}
\end{sectionbox}

\begin{sectionbox}
\subsection{Beispiel: Sliding Window}
\begin{lstlisting}[language=Python]
def main():
  text = input()

  map = {'a':0, 'b':0, 'c':0}
  bestl = -1
  bestr = len(text)
  l=0
  r=-1
  num=0
  while r < len(text):
    if num == 3 and bestr-bestl > r-l:
      bestl = l
      bestr = r
    if num >= 3:
      x = text[l]
      if x in map:
        xc = map[x]
        xc -= 1
        map[x]=xc
        if xc == 0:
            num -= 1
      l += 1
    else:
      r += 1
      if r < len(text):
        x = text[r]
        if x in map:
            xc = map[x]
            xc += 1
            map[x] = xc
            if xc == 1:
              num += 1
  if bestl == -1:
    print(text,"does not contain a,b AND c.")
  else:
    print("contains a,b,c between",bestl,"and",bestr)
\end{lstlisting}
\end{sectionbox}

\begin{sectionbox}
\subsection{Beispiel: Palindrome Checker}
\begin{lstlisting}[language=Python]
def isPalindrome(word):
  for i in range(0, len(word)//2):
    if word[i] != word[-1 -i]:
      return False
  return True
    
def main():
  again, word = True, input("Enter a word: ")
  while again:
    if isPalindrome(word):
      cprint(word + ' is a palindrome!')
    else:
      cprint(word + ' is not a palindrome')
    word = input("Enter a word (or just <ENTER> to stop): ")
    again = len(word) > 0
\end{lstlisting}\vspace{-6px}
\end{sectionbox}